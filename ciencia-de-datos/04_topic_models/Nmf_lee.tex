\documentclass{beamer}
\usepackage[utf8]{inputenc}
\usepackage{amsmath, amssymb, bm}
\usepackage{graphicx}
\usepackage{booktabs}
\usepackage{tikz}

%\usetheme{Madrid}

\title{Método de Lee y Seung para la Descomposición No Negativa de Matrices (NMF)}
\author{Presentado por ChatGPT}
\date{\today}

\begin{document}

\frame{\titlepage}


% Slide 1: Introducción a NMF
\begin{frame}{¿Qué es NMF?}
\begin{itemize}
    \item Dada una matriz no negativa \( V \in \mathbb{R}^{m \times n} \), se busca una aproximación:
    \[
    V \approx WH
    \]
    \item Donde:
    \begin{itemize}
        \item \( W \in \mathbb{R}^{m \times r} \): matriz base
        \item \( H \in \mathbb{R}^{r \times n} \): matriz de coeficientes
        \item \( r \ll \min(m, n) \): número de componentes latentes
    \end{itemize}
    \item Las tres matrices tienen entradas no negativas.
\end{itemize}
\end{frame}

% Slide 2: Objetivo de la factorización
\begin{frame}{Objetivo de la factorización}
\begin{itemize}
    \item Minimizar la diferencia entre \( V \) y \( WH \):
    \[
    \min_{W, H \geq 0} \quad D(V \Vert WH)
    \]
    \item Funciones de costo comunes:
    \begin{itemize}
        \item Error cuadrático: \( \| V - WH \|_F^2 \)
        \item Divergencia KL: \( D(V \Vert WH) \)
    \end{itemize}
    \item Restricción: \( W, H \geq 0 \)
\end{itemize}
\end{frame}

% Slide 3: Algoritmo de Lee y Seung
\begin{frame}{Método de Lee y Seung}
\begin{itemize}
    \item Propuesto en 1999 (versión mejorada en 2001).
    \item Basado en actualizaciones multiplicativas.
    \item Asegura que las matrices \( W \) y \( H \) sigan siendo no negativas.
    \item Dos versiones: para el error cuadrático y para la divergencia KL.
\end{itemize}
\end{frame}

% Slide 4: Caso 1 - Error cuadrático
\begin{frame}{Actualizaciones para el error cuadrático}
\begin{itemize}
    \item Función de pérdida:
    \[
    \mathcal{L} = \frac{1}{2} \| V - WH \|_F^2
    \]
    \item Reglas de actualización:
    \[
    H \leftarrow H \circ \frac{W^\top V}{W^\top W H}
    \]
    \[
    W \leftarrow W \circ \frac{V H^\top}{W H H^\top}
    \]
    \item Donde \( \circ \) denota multiplicación elemento a elemento.
\end{itemize}
\end{frame}

% Slide 5: Caso 2 - Divergencia KL
\begin{frame}{Actualizaciones para la divergencia KL}
\begin{itemize}
    \item Función de pérdida:
    \[
    D(V \Vert WH) = \sum_{i,j} \left( V_{ij} \log \frac{V_{ij}}{(WH)_{ij}} - V_{ij} + (WH)_{ij} \right)
    \]
    \item Reglas de actualización:
    \[
    H_{ak} \leftarrow H_{ak} \cdot \frac{\sum_i W_{ia} \frac{V_{ik}}{(WH)_{ik}}}{\sum_i W_{ia}}
    \]
    \[
    W_{ia} \leftarrow W_{ia} \cdot \frac{\sum_k H_{ak} \frac{V_{ik}}{(WH)_{ik}}}{\sum_k H_{ak}}
    \]
\end{itemize}
\end{frame}


%%%%%%%%
% Introducción
\begin{frame}{¿Qué es NMF?}
    \textbf{Descomposición No Negativa de Matrices (NMF)} busca factorizar una matriz \( V \in \mathbb{R}^{m \times n} \), con \( V \geq 0 \), en el producto de dos matrices no negativas:
    \[
        V \approx WH, \quad W \in \mathbb{R}^{m \times r}, \quad H \in \mathbb{R}^{r \times n}
    \]
    \pause
    \textbf{Aplicaciones:}
    \begin{itemize}
        \item Minería de texto
        \item Reconocimiento de patrones
        \item Procesamiento de imágenes y bioinformática
    \end{itemize}
\end{frame}

% Método de Lee y Seung
\begin{frame}{Método de Lee y Seung (2001)}
    Lee y Seung propusieron reglas de actualización multiplicativas para minimizar:
    \[
        \min_{W,H \geq 0} \| V - WH \|_F^2
    \]
    Las reglas de actualización son:
    \begin{align*}
        H &\leftarrow H \circ \frac{W^\top V}{W^\top WH} \\
        W &\leftarrow W \circ \frac{V H^\top}{WHH^\top}
    \end{align*}
    donde \( \circ \) denota el producto elemento a elemento (Hadamard).
\end{frame}

% Datos iniciales
\begin{frame}{Ejemplo numérico: datos iniciales}
\begin{itemize}
    \item Matriz original:
    \[
        V = \begin{bmatrix}
            5 & 3 & 1 \\
            2 & 3 & 4
        \end{bmatrix}
    \]
    \item Inicialización aleatoria:
    \[
        W^{(0)} = \begin{bmatrix}
            0.6 & 0.8 \\
            0.3 & 0.4
        \end{bmatrix}, \quad
        H^{(0)} = \begin{bmatrix}
            1 & 0.5 & 0.2 \\
            0.3 & 0.8 & 0.6
        \end{bmatrix}
    \]
\end{itemize}
\end{frame}

% Iteración 1 - actualizar H
\begin{frame}{Iteración 1: actualización de \( H \)}
\textbf{Paso 1: Calcular} \( H^{(1)} = H \circ \frac{W^\top V}{W^\top WH} \)

\[
W^\top V =
\begin{bmatrix}
4.2 & 2.7 & 1.8 \\
4.8 & 3.6 & 2.4
\end{bmatrix}, \quad
W^\top W H =
\begin{bmatrix}
0.63 & 0.705 & 0.45 \\
0.84 & 0.94 & 0.60
\end{bmatrix}
\]

\[
H^{(1)} =
\begin{bmatrix}
6.67 & 1.91 & 0.80 \\
1.71 & 3.06 & 2.40
\end{bmatrix}
\]
\end{frame}

% Iteración 2 - actualizar H
\begin{frame}{Iteración 2: actualización de \( H \)}
\textbf{Paso 2: Usamos \( H^{(1)} \) para una nueva actualización}

\[
W^\top W H^{(1)} \approx
\begin{bmatrix}
4.03 & 2.70 & 1.80 \\
5.37 & 3.59 & 2.40
\end{bmatrix}
\quad
W^\top V =
\begin{bmatrix}
4.2 & 2.7 & 1.8 \\
4.8 & 3.6 & 2.4
\end{bmatrix}
\]

\[
H^{(2)} \approx
\begin{bmatrix}
6.95 & 1.91 & 0.80 \\
1.53 & 3.07 & 2.40
\end{bmatrix}
\]
\end{frame}

% Iteración 2 - actualizar W
\begin{frame}{Iteración 2: actualización de \( W \)}
\textbf{Paso 3: actualizar \( W \)}

\[
H^{(2)} =
\begin{bmatrix}
6.95 & 1.91 & 0.80 \\
1.53 & 3.07 & 2.40
\end{bmatrix}
\Rightarrow
H H^\top \approx
\begin{bmatrix}
52.6 & 18.41 \\
18.41 & 17.52
\end{bmatrix}
\]

\[
V H^\top \approx
\begin{bmatrix}
41.28 & 19.26 \\
22.83 & 21.87
\end{bmatrix}
\]

\[
W^{(2)} = W \circ \frac{V H^\top}{W H H^\top}
\]
(operación elemento a elemento)
\end{frame}

% Conclusión
\begin{frame}{Conclusión}
\begin{itemize}
    \item El método de Lee y Seung ofrece una forma eficiente y sencilla de resolver el problema NMF.
    \item Sus actualizaciones multiplicativas preservan la no negatividad.
    \item El ejemplo con dos iteraciones mostró cómo evoluciona la aproximación.
\end{itemize}
\pause
\textbf{¿Ventajas?}
\begin{itemize}
    \item Fácil implementación.
    \item Estabilidad garantizada (convergencia del error).
\end{itemize}
\end{frame}

\end{document}