\documentclass{beamer}
\usepackage[utf8]{inputenc}
\usepackage{amsmath}

\title{Ejemplo de NMF con algoritmo de Lee y Seung}
\author{JFZO - EST297}
\date{2025}

\begin{document}

\frame{\titlepage}


% Slide 1: Introducción a NMF
\begin{frame}{¿Qué es NMF?}
\begin{itemize}
    \item Dada una matriz no negativa \( V \in \mathbb{R}^{m \times n} \), se busca una aproximación:
    \[
    V \approx WH
    \]
    \item Donde:
    \begin{itemize}
        \item \( W \in \mathbb{R}^{m \times r} \): matriz base
        \item \( H \in \mathbb{R}^{r \times n} \): matriz de coeficientes
        \item \( r \ll \min(m, n) \): número de componentes latentes
    \end{itemize}
    \item Las tres matrices tienen entradas no negativas.
\end{itemize}
\end{frame}

% Slide 2: Objetivo de la factorización
\begin{frame}{Objetivo de la factorización}
\begin{itemize}
    \item Minimizar la diferencia entre \( V \) y \( WH \):
    \[
    \min_{W, H \geq 0} \quad D(V \Vert WH)
    \]
    \item Funciones de costo comunes:
    \begin{itemize}
        \item Error cuadrático: \( \| V - WH \|_F^2 \)
        \item Divergencia KL: \( D(V \Vert WH) \)
    \end{itemize}
    \item Restricción: \( W, H \geq 0 \)
\end{itemize}
\end{frame}

% Slide 3: Algoritmo de Lee y Seung
\begin{frame}{Método de Lee y Seung}
\begin{itemize}
    \item Propuesto en 1999 (versión mejorada en 2001).
    \item Basado en actualizaciones multiplicativas.
    \item Asegura que las matrices \( W \) y \( H \) sigan siendo no negativas.
    \item Dos versiones: para el error cuadrático y para la divergencia KL.
\end{itemize}
\end{frame}

% Slide 4: Caso 1 - Error cuadrático
\begin{frame}{Actualizaciones para el error cuadrático}
\begin{itemize}
    \item Función de pérdida:
    \[
    \mathcal{L} = \frac{1}{2} \| V - WH \|_F^2
    \]
    \item Reglas de actualización:
    \[
    H \leftarrow H \circ \frac{W^\top V}{W^\top W H}
    \]
    \[
    W \leftarrow W \circ \frac{V H^\top}{W H H^\top}
    \]
    \item Donde \( \circ \) denota multiplicación elemento a elemento.
\end{itemize}
\end{frame}

% Slide 5: Caso 2 - Divergencia KL
\begin{frame}{Actualizaciones para la divergencia KL}
\begin{itemize}
    \item Función de pérdida:
    \[
    D(V \Vert WH) = \sum_{i,j} \left( V_{ij} \log \frac{V_{ij}}{(WH)_{ij}} - V_{ij} + (WH)_{ij} \right)
    \]
    \item Reglas de actualización:
    \[
    H_{ak} \leftarrow H_{ak} \cdot \frac{\sum_i W_{ia} \frac{V_{ik}}{(WH)_{ik}}}{\sum_i W_{ia}}
    \]
    \[
    W_{ia} \leftarrow W_{ia} \cdot \frac{\sum_k H_{ak} \frac{V_{ik}}{(WH)_{ik}}}{\sum_k H_{ak}}
    \]
\end{itemize}
\end{frame}


%%%%%% nuevo ejemplo mas detallado

\begin{frame}{Matriz original y objetivo}
Matriz original \( V \):

\[
V = \begin{bmatrix}
5 & 3 \\
3 & 2 \\
4 & 1
\end{bmatrix}
\]

Factorizar \( V \approx W H \), con \( W \in \mathbb{R}^{3 \times 2}_{\geq 0} \), \( H \in \mathbb{R}^{2 \times 2}_{\geq 0} \).
\end{frame}

\begin{frame}{Inicialización}
Matrices iniciales:

\[
W^{(0)} = \begin{bmatrix}
1 & 0.5 \\
0.5 & 1 \\
1 & 1
\end{bmatrix}
\quad , \quad
H^{(0)} = \begin{bmatrix}
1 & 2 \\
1 & 1
\end{bmatrix}
\]
\end{frame}

\begin{frame}{Iteración 1: Actualización de \( H \)}
\[
H_{ij} \leftarrow H_{ij} \times \frac{(W^T V)_{ij}}{(W^T W H)_{ij}}
\]

\[
W^T V = \begin{bmatrix}
9.5 & 4.0 \\
9.5 & 4.5
\end{bmatrix}
\quad , \quad
W^T W H^{(0)} = \begin{bmatrix}
4.25 & 6.5 \\
4.25 & 6.25
\end{bmatrix}
\]

\[
H^{(1)} = \begin{bmatrix}
2.235 & 1.231 \\
2.235 & 0.72
\end{bmatrix}
\]
\end{frame}

\begin{frame}{Iteración 1: Actualización de \( W \)}
\[
W_{ij} \leftarrow W_{ij} \times \frac{(V H^T)_{ij}}{(W H H^T)_{ij}}
\]

\[
V H^{(1)T} = \begin{bmatrix}
14.88 & 13.365 \\
9.21 & 8.085 \\
10.171 & 9.16
\end{bmatrix}
\quad , \quad
W^{(0)} H^{(1)} H^{(1)T} = \begin{bmatrix}
9.54 & 8.49 \\
8.06 & 8.4 \\
12.6 & 11.26
\end{bmatrix}
\]

\[
W^{(1)} = \begin{bmatrix}
1.56 & 0.79 \\
0.57 & 0.96 \\
0.81 & 0.81
\end{bmatrix}
\]
\end{frame}

\begin{frame}{Iteración 2: Actualización de \( H \)}
\[
W^{(1)T} V = \begin{bmatrix}
13.89 & 6.75 \\
11.37 & 5.04
\end{bmatrix}
\quad , \quad
W^{(1)T} W^{(1)} H^{(1)} = \begin{bmatrix}
12.98 & 5.76 \\
9.94 & 4.21
\end{bmatrix}
\]

\[
H^{(2)} = \begin{bmatrix}
2.39 & 1.44 \\
2.56 & 0.86
\end{bmatrix}
\]
\end{frame}

\begin{frame}{Iteración 2: Actualización de \( W \)}
\[
V H^{(2)T} = \begin{bmatrix}
20.35 & 17.45 \\
12.87 & 11.12 \\
11.98 & 10.17
\end{bmatrix}
\quad , \quad
W^{(1)} H^{(2)} H^{(2)T} = \begin{bmatrix}
18.1 & 15.7 \\
14.9 & 13.2 \\
15.9 & 14.2
\end{bmatrix}
\]

\[
W^{(2)} = \begin{bmatrix}
1.75 & 0.88 \\
0.49 & 0.81 \\
0.61 & 0.58
\end{bmatrix}
\]
\end{frame}

\begin{frame}{Iteración 3: Actualización de \( H \)}
\[
W^{(2)T} V = \begin{bmatrix}
14.6 & 7.1 \\
11.4 & 5.2
\end{bmatrix}
\quad , \quad
W^{(2)T} W^{(2)} H^{(2)} = \begin{bmatrix}
13.9 & 6.1 \\
10.7 & 4.7
\end{bmatrix}
\]

\[
H^{(3)} = \begin{bmatrix}
2.51 & 1.68 \\
2.73 & 0.95
\end{bmatrix}
\]
\end{frame}

\begin{frame}{Iteración 3: Actualización de \( W \)}
\[
V H^{(3)T} = \begin{bmatrix}
21.2 & 18.1 \\
13.2 & 11.5 \\
12.1 & 10.5
\end{bmatrix}
\quad , \quad
W^{(2)} H^{(3)} H^{(3)T} = \begin{bmatrix}
19.3 & 16.8 \\
15.8 & 14.0 \\
16.8 & 15.0
\end{bmatrix}
\]

\[
W^{(3)} = \begin{bmatrix}
1.92 & 0.95 \\
0.41 & 0.67 \\
0.44 & 0.41
\end{bmatrix}
\]
\end{frame}

\begin{frame}{Iteración 4: Actualización de \( H \) y resultado final}
\[
W^{(3)T} V = \begin{bmatrix}
15.1 & 7.3 \\
11.2 & 5.0
\end{bmatrix}
\quad , \quad
W^{(3)T} W^{(3)} H^{(3)} = \begin{bmatrix}
14.6 & 6.4 \\
10.9 & 4.7
\end{bmatrix}
\]

\[
H^{(4)} = \begin{bmatrix}
2.60 & 1.92 \\
2.80 & 1.01
\end{bmatrix}
\]

Resultado final aproximado:

\[
W^{(4)} \approx \begin{bmatrix}
1.92 & 0.95 \\
0.41 & 0.67 \\
0.44 & 0.41
\end{bmatrix}
\quad , \quad
H^{(4)} \approx \begin{bmatrix}
2.60 & 1.92 \\
2.80 & 1.01
\end{bmatrix}
\]

y

\[
W^{(4)} H^{(4)} \approx V
\]
\end{frame}
\end{document}
